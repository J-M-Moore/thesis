%%%%%%%%%%%%%%%%%%%%%%%%%%%%%%%%%%%%%%%%%%%%%%%%%%%%%%%%%%%%%%%%%%%%%%%%%%%%%%%%%%%%%%%
\begin{table}[H]
  \begin{center}
{\fontsize{7pt}{7pt}\selectfont
%\begin{table}[t]
  \centering
   \renewcommand{\arraystretch}{1.5}
   \setlength{\tabcolsep}{2pt}
   \begin{tabular}{lccccc|c}
     \toprule \textbf{Experiment}     & $\bf{\sqrt{s}}$\textbf{(TeV)}
     &   \textbf{Observable} & $\mathcal{L}$ (fb${}^{-1}$) & $\mathbf{n_{\rm dat}}$ & \textbf{Ref.}   
    &  \textbf{New (SMEFT fits)} \\
     \toprule
           ATLAS/CMS comb.
      & 8
      & $F_0, F_L$
      & $20$
      & 2
      & \cite{Aad:2020jvx}
      &  \\
 \midrule
       ATLAS
      & 13
      & $F_0, F_L$
      & $139$
      & 2
      & \cite{ATLAS:2022bdg}
      & $\checkmark$     \\
\bottomrule
   \end{tabular}
   \vspace{0.3cm}
  \caption{\small Same as Table~\ref{tab:input_datasets_toppair} for
    the $W$-helicity fraction measurements.
    %
    These helicity fractions are PDF-independent and hence
    are only relevant in constraining the EFT coefficients.
    \label{tab:whelicities}
   \label{tab:input_datasets_helicityfractions}
}
}
\end{center}
\end{table}
%%%%%%%%%%%%%%%%%%%%%%%%%%%%%%%%%%%%%%%%%%%%%%%%%%%%%%%%%%%%%%%%%%%%%%%%%%%%%%%%
